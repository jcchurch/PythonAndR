\documentclass[letterpaper,10pt]{article}
\usepackage[top=1in,right=1in,left=1in,bottom=0.5in]{geometry}
\begin{document}

\noindent

\begin{center}
\begin{bfseries}
Data Analysis using Python and R

Winter Intersession 2012

Syllabus
\end{bfseries}
\end{center}

\begin{tabular}{|p{2in}|p{4in}|} \hline
Class Time&Monday, Tuesday, Wednesday, Thursday, Friday, 1:00 PM - 4:45 PM\\ \hline
Classroom&Weir 235\\ \hline
Instructor&James Church\\ \hline
Personal Webpage&\verb+http://home.olemiss.edu/~jcchurch/+\\ \hline
Course Web page&\verb+https://github.com/jcchurch/PythonAndR+\\ \hline
Office&Weir 232\\ \hline
Office Hours&Monday through Friday, any time in the morning, or by appointment\\ \hline
Email&jcchurch@olemiss.edu\\ \hline
Required Text&Data Analysis with Open Source Tools, by Philip K. Janert, ISBN: 9780596802356\\ \hline
Course Description&This course will have a dual focus. The first focus is about data analysis from a scientific perspective. We will be discussing how to sample and analyze data (both small and large) and build models based on our assumptions. We will be using the tools Python and R. The second focus of this course is to teach the student all of the syntax behind Python and R.\\ \hline
Course Pre-requisites& CSCI 211 (Computer Science III)\\ \hline
Final Examination&Monday, June 22 at 4 p.m.\\ \hline
Grading&
The overall grade for this course will be determined according to the following weights:
Homework (60\%), Test 1 (10\%), Test 2 (10\%), Final Exam (20\%)
\\ \hline
Grading Scale for this class& $A (>= 90), B (>= 80, <90), C (>= 70, <80), D (>= 60, <70), F (<60)$ \\ \hline
Professional Conduct Policy& All students in CSCI 390 Section 1 are expected to conduct themselves in a
professional manner according to the Honor Code of the School of Engineering,
the Information Technology Appropriate Use Policy, the M Book, and any other
relevant policies.
``The Honor Code shall apply to all students, both undergraduate and graduate,
registered in and/or seeking degrees through the School of Engineering. The
Honor Code shall be understood to apply to all academic areas of the School
such as examinations, quizzes, laboratory reports, themes, computer programs,
homework, and other possible assignments. Only that work explicitly identified by
the class instructor not to be under the Honor Code is excluded. The intent of the
Honor Code is to recognize professional conduct and, thus, it shall be deemed a
violation of the Honor Code to knowingly deceive, copy, paraphrase, or otherwise
misrepresent your work in a manner inconsistent with professional conduct.''\\ \hline
Student Disabilities Services Statement& ``It is the responsibility of any student with a disability who requests a reasonable accommodation to contact the Office of Disability Services (915-7128). Contact will then be made by that office through the student to the instructor of this class. The instructor will then be happy to work with the student so that a reasonable accommodation of any disability can be made.''\\ \hline
Notes on Assignments&Some assignments will require you to write source code and send that code to the instructor. Please send your assignments from a real e-mail account. I'll be sending your feedback through that account. There are 7 assignments total. All assignments are due at 8 AM of the due date (this is so that I may grade them and hand them back the same day).\\ \hline
\end{tabular}

\end{document}
